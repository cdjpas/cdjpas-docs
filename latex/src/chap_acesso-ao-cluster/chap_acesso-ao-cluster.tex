\chapter{Acesso ao Cluster}
\label{chap:acesso-ao-cluster}

\section{Solicitação de Acesso}
\label{sec:solicitacao-de-acesso}
% Procedimentos para solicitar uma conta de usuário.

Para ser um usuário do cluster, você precisa atender um dos seguintes pré-requisitos:

\itmz{
    \item Ser aluno(a) do programa de pós-graduação em Astronomia do ON, ou
      pós-doutorando na COAST, ou pesquisador membro da COAST;
    \item Ser membro externo ao ON, mas colaborador(a) envolvido(a) em algum projeto de pesquisa da qual a COAST é participante.
}

O acesso ao cluster CDJPAS não é garantido automaticamente através da Conta de Serviços Internos (CSI) que o usuário utiliza para acessar os diversos serviços fornecidos pela DITIN (VPN, LDAP, intranet, suporte, etc.).
Para obter acesso ao cluster, você deve solicitar a abertura de conta abrindo um chamado através do Suporte ON\footnote{\url{\URLSuporteON}.}, indicando \textbf{\cdjpasadmin} como responsável por atender o chamado.
As seguintes informações devem ser fornecidas no chamado:

\itmz{
    \item Nome completo;
    \item Posição acadêmica (mestrando, doutorando, pós-doc ou pesquisador);
    \item Área ou grupo de pesquisa;
    \item Nome do orientador ou supervisor (se você for aluno ou pós-doc);
    \item Projeto envolvido.
}

As solicitações feitas para pessoas externas ao ON devem ser realizadas por membros da COAST e incluir a instituição da qual a pessoa faz parte.



\section{Acessando o cluster}
\label{sec:acessando-o-cluster}
% Métodos de login (SSH, VPN, etc.) e autenticação multifatorial, se aplicável.

O acesso ao cluster se dá via nó de \textit{login}, que possui o endereço \hostname{cdjpas.on.br}.
Este endereço só pode ser acessado por quem está conectado à rede interna do ON, o que pode ser feito a partir de qualquer computador localizado dentro do campus.
Para acessá-lo estando fora do campus, o usuário deve se conectar primeiro à rede do ON via conta VPN\footnote{Acesse \url{\URLIntranetONVPN} para solicitar um certificado VPN. Note que é necessário já possuir uma conta CSI para fazer essa solicitação.}.
É no nó de \textit{login} que os usuários devem submeter seus jobs ao gerenciador de filas Slurm.
Este, por sua vez, irá delegar o trabalho aos nós de computação.

% Acesso a partir do Linux ou OS X



\section{Se conectando com senha}
\label{sec:se-conectando-com-senha}
% Instruções detalhadas para acesso remoto usando diferentes sistemas operacionais (Windows, MacOS, Linux).

O acesso remoto ao cluster se dá via conexão SSH ao nó de \textit{login}.
Isto pode ser feito com o seguinte comando:

\begin{terminal}
    ssh usuario@cdjpas.on.br
\end{terminal}

\noindent onde \textverb{usuario} é o seu nome de usuário no cluster.
Apenas sua senha será exigida após este comando.
Se necessário, digite \command{man ssh} para obter mais informações sobre o comando \command{ssh}.
Esse método é ideal para usuários que não precisam acessar o cluster com frequência ou que preferem uma autenticação simples.

\begin{warning}
    O usuário deve evitar o uso da opção \textverb{-X} (\command{ssh -X ...}), que habilita o encaminhamento X11 (\textit{X11 Forwarding}), uma vez que o cluster não possui ambiente gráfico instalado.
\end{warning}



\section{Se conectando com chaves SSH}
\label{sec:se-conectando-com-chaves-ssh}

Muitas vezes, digitar uma senha sempre que se deseja realizar uma conexão via SSH com algum servidor remoto pode ser um processo bem tedioso, além de expor o usuário a riscos desnecessários relacionados ao gerenciamento de senhas.
Um método mais recomendado em conexões deste tipo é a utilização de um par de chaves SSH\footnote{Você acessar \url{\URLSSHKeys} para ver uma descrição detalhada de como criar suas chaves e utilizá-las no cluster.}, que elimina a necessidade de senha e torna a conexão mais direta e menos arriscada.


% Criando o par de chaves
\subsection{Passos para configurar chaves SSH}
\label{subsec:passos-para-configurar-chaves-SSH}

% Copiando as chaves para o nó de login
\begin{enumerate}
    \item Gere um par de chaves SSH no seu computador local (se você ainda não tiver uma):
    \begin{terminal}
        ssh-keygen -t ed25519 -b 4096
    \end{terminal}
        Siga as instruções para salvar a chave (por exemplo, em \filename{~/.ssh/id_ed25519}) e definir uma senha opcional para a chave.

    \item Copie a chave pública para o CDJPAS:
    \begin{terminal}
        ssh-copy-id usuario@cdjpas.on.br
    \end{terminal}
        Insira sua senha quando solicitado. Isso adicionará sua chave pública ao arquivo \filename{~/.ssh/authorized_keys} no cluster.

    \item Agora, você pode se conectar ao cluster sem precisar digitar a senha:
    \begin{terminal}
        ssh usuario@cdjpas.on.br
    \end{terminal}
\end{enumerate}

\subsection{Exemplo de uso}
\label{subsec:exemplo-de-uso}

Se você gerou uma chave SSH chamada \filename{~/.ssh/id_ed25519}, pode usá-la diretamente no comando SSH:
\begin{terminal}
    ssh -i ~/.ssh/id_ed25519 usuario@cdjpas.on.br
\end{terminal}



\section{Se conectando através do arquivo de configuração do SSH}
\label{sec:se-conectando-atraves-do-arquivo-de-configuracao-do-SSH}

Outro método recomendado para se conectar ao cluster é o uso de um arquivo de configuração SSH para simplificar as conexões. O arquivo de configuração do SSH, localizado em \filename{~/.ssh/config}, permite definir configurações personalizadas para diferentes hosts, incluindo o CDJPAS. Esse método é especialmente útil para quem acessa múltiplos clusters ou precisa de configurações específicas.


\subsection{Exemplo de configuração}
\label{subsec:exemplo-de-configuracao}

\begin{enumerate}
    \item Abra ou crie o arquivo \filename{~/.ssh/config} no seu computador local:
    \begin{terminal}
        nano ~/.ssh/config
    \end{terminal}

    \item Adicione uma entrada para o CDJPAS:
    \begin{code}{linuxconfig}
        Host cdjpas
            hostname cdjpas.on.br
            user usuario
            IdentityFile ~/.ssh/id_ed25519
    \end{code}

    \item Salve o arquivo e feche o editor.

    \item Agora, você pode se conectar ao CDJPAS usando apenas o alias definido:
    \begin{terminal}
        ssh cdjpas
    \end{terminal}
\end{enumerate}

Como vantagem, esse método torna desnecessário a digitação do nome de usuário e/ou do endereço completo toda vez que o usuário quiser se conectar ao cluster. Além disso, ele facilita o gerenciamento de múltiplas chaves SSH e configurações. Por fim, a autenticação do usuário no cluster via chaves SSH é mais segura do que a autenticação com senha, uma vez que as chaves são encriptadas e o usuário não precisa enviar sua senha via rede.

\begin{warning}
    Verifique se o arquivo \filename{~/.ssh/config} possui no mínimo permissão para leitura e escrita para o seu nome de usuário.
    Você pode usar o comando \command{chmod 600 ~/.ssh/config} caso não tenha certeza.
\end{warning}

% Acesso a partir do Windows
