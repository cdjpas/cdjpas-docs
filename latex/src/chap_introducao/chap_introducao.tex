\chapter{Introdução}
\label{chap:introducao}

\section{Objetivo do manual}
\label{sec:objetivo-do-manual}
% Breve descrição do propósito do manual.

Esta é a documentação do CDJPAS, o cluster para computação de alto desempenho da Coordenação de Astronomia e Astrofísica do Observatório Nacional/MCTI (COAST). O propósito principal deste documento é descrever a estrutura e o funcionamento do cluster, além de fornecer todas as informações necessárias para que seus usuários consigam executar as tarefas envolvidas com seus trabalhos científicos.

\section{Público-Alvo}
\label{sec:publico-alvo}
% Descrição dos usuários previstos: pesquisadores, cientistas, estudantes de pós-graduação, etc.

Este manual tem como público-alvo os atuais usuários do cluster CDJPAS, assim como possíveis interessados em usufruir do cluster em algum momento. Os usuários devem ser participantes dos seguintes grupos:
\itmz{
    \item Estudantes de pós-graduação, pós-doutorandos e pesquisadores da COAST;
    \item Membros do levantamento astronômico Javalambre Physics of the Accelerating Universe Astrophysical Survey (J-PAS)\footnote{\url{\URLJPAS}.}.
}

\section{Estrutura do Manual}
\label{sec:estrutura-do-manual}
% Visão geral da organização do conteúdo do manual.

O conteúdo deste manual está organizado da seguinte forma:

\itmz{
    \item Capítulo~\ref{chap:visao-geral-do-cdjpas}: apresenta uma visão geral da estrutura e organização física e lógica do CDJPAS, tanto no que diz respeito ao hardware instalado como também ao ecossistema de software nele contido.
    \item Capítulo~\ref{chap:acesso-ao-cluster}: apresenta os métodos de acesso ao cluster, seja a partir do campus do Observatório Nacional como também de fora dele.
    \item Capítulo~\ref{chap:uso-do-cluster}: mostra as ferramentas e métodos necessários para os usuários executarem suas tarefas no cluster.
    \item Capítulo~\ref{chap:gestao-de-dados}: apresenta as práticas envolvidas com a gestão dos dados científicos dos usuários no cluster, tais como armazenamento, transferência, etc.
    \item Capítulo~\ref{chap:software}: fornece informações sobre o ambiente de desenvolvimento instalado no cluster, bem como sugestões sobre otimização visando melhorar a performance das tarefas que os usuários executam nele.
    \item Capítulo~\ref{chap:politicas-e-normas}: apresenta as políticas adotadas para a administração do cluster e as regras que os usuários devem seguir em obediência a essas políticas.
    \item Capítulos~\ref{chap:suporte-e-recursos} e~\ref{chap:solucoes-de-problemas-comuns}: fornecem informações sobre obtenção de suporte e possíveis soluções dos problemas mais comuns relatados pelos usuários do cluster.
    \item Apêndices~\ref{app:tutoriais} e~\ref{app:exemplos-de-scripts}: apresenta tutoriais e exemplos de scripts para algumas tarefas comuns em ambientes HPC e que os usuários podem achar úteis dentro do ambiente do CDJPAS.
}
