%##############################################################################
%##############################################################################
% CONFIGURATION OF THE DOCUMENT
%##############################################################################
%##############################################################################


%##############################################################################
%=== LAYOUT ===================================================================
%##############################################################################
% Page margins
% Some tips based on the golden ratio (see https://goo.gl/PpwfYX):
%   - Top margins are about half of bottom margin
%   - Outside margins are about 2/3 rds of bottom margin
%   - Inside margins are about 1/3 rds of bottom margin
\geometry{
    paper=a4paper,
    %bottom=6cm,inner=2cm,top=3cm,outer=4cm,  % With footnote in the text
    bottom=4.5cm,inner=1.5cm,top=2.25cm,outer=3cm,  % No footnote in the text
}

% Spacing between lines
\linespread{1.1}

% Chapter's style
%\chapterstyle{hangnum}
% Style of sections, subsections, and subsubsections
\setsecheadstyle{\Large\bfseries\sffamily\raggedright}
\setsubsecheadstyle{\large\bfseries\sffamily\raggedright}
\setsubsubsecheadstyle{\bfseries\sffamily\raggedright}
% To make sections be numbered 1, 2, ... instead of 0.1, 0.2, ...
%\counterwithout{section}{chapter}

% Style of pages numbering
\makepagestyle{myheadings}
\makeevenhead{myheadings}{% {pagestyle}{left}{center}{right}
    \sffamily\thepage\hspace{0.5cm}\textbf{\scriptsize\leftmark}\hrule%
}{}{}
\makeoddhead{myheadings}{}{}{% {pagestyle}{left}{center}{right}
    \sffamily\textbf{\scriptsize\rightmark}\hspace{0.5cm}\thepage\hrule%
}
%\makeevenfoot{myheadings}{}{}{}  % {pagestyle}{left}{center}{right}
%\makeoddfoot{myheadings}{}{}{}  % {pagestyle}{left}{center}{right}
\pagestyle{myheadings}

% Chapters, sections, and subsections are numbered
\maxsecnumdepth{subsection}
% Chapters, sections, and subsections are in the Table of Contents
\maxtocdepth{subsection}

% Reset length names that control space above and below rules in booktabs.
% See https://tex.stackexchange.com/a/652375.
\setlength{\aboverulesep}{0pt}
\setlength{\belowrulesep}{0pt}
\renewcommand{\arraystretch}{1.5}

% This defines a latex counter to shadow the `\newcount' allocated counter defined by `xcolor'.
% See https://tex.stackexchange.com/a/297357.
\newcounter{tblerows}
\expandafter\let\csname c@tblerows\endcsname\rownum


%##############################################################################
%=== LINKS ====================================================================
%##############################################################################
\hypersetup{
    %pdftitle={\@title - \doctype},
    %pdfauthor={\authorshort},
    pdfkeywords={\keywords},
    breaklinks=true,
    bookmarksopen=true,
    bookmarksopenlevel=2,
    bookmarksnumbered,
    colorlinks=true,
    citecolor=MidnightBlue,
    linkcolor=Maroon,
    urlcolor=MidnightBlue,
    pdfpagemode=UseOutlines,
    pdfstartview={FitH},
    breaklinks=true,
    linktocpage=true
}


%##############################################################################
%=== SOME STYLES ==============================================================
%##############################################################################
% Chapter's title
\titleformat
	{\chapter}
	[hang]
	{\Huge\bfseries\sffamily}
	{\MakeUppercase{\thechapter}}
	{20pt}
	{\huge\sffamily}[\vspace{0.5cm} \titlerule]
% This changes the spacing "before" the chapter's title (the second lenght
% argument) to 0.
\titlespacing*{\chapter}{0pt}{3cm}{1.5cm}

% Turn footnote numbers into symbols
\renewcommand{\thefootnote}{\fnsymbol{footnote}}

% Makes to width of the inserted todonotes to fit inside the available space in
% the margin. This should be equal to the value of the textwidth option of the
% todonotes package.
\setlength{\marginparwidth}{2.5cm}

% Style for versioning based on the date.
% Based on https://latex.org/forum/viewtopic.php?p=119342#p119342.
\DTMnewdatestyle{versionbydate}{
    \renewcommand*\DTMdisplaydate[4]{%
        \DTMtwodigits{##1}.\DTMtwodigits{##2}.\DTMtwodigits{##3}%
    }
    \renewcommand*\DTMDisplaydate{\DTMdisplaydate}
}

% Save the output of the \today command in the variable \docversion.
% See https://tex.stackexchange.com/a/435038
\DTMsetdatestyle{versionbydate}
\newcommand{\tmpvalue}{\today}
\edef\docversion{\tmpvalue}
\DTMsetdatestyle{default}

% Style for paragraph titles (Memoir manual, section 6.6)
\setparaheadstyle{\sffamily\bfseries}


%##############################################################################
%=== (RE)DEFINITIONS OF SOME COMMANDS OF LaTeX ================================
%##############################################################################
\addto\captionsbrazilian{\renewcommand{\contentsname}{\sffamily\textbf{Conteúdo}}}


%##############################################################################
%=== REFERENCES FILES =========================================================
%##############################################################################
\addbibresource{src/pre/pre_ref.bib}
\ExecuteBibliographyOptions{doi=false}
%\newbibmacro{string+doi}[1]{%
%    \iffieldundef{doi}{#1}{\href{http://dx.doi.org/\thefield{doi}}{#1}}}
%\DeclareFieldFormat{title}{\usebibmacro{string+doi}{\mkbibemph{#1}}}
%\DeclareFieldFormat[article]{title}{\usebibmacro{string+doi}{\mkbibquote{#1}}}
\DeclareFieldFormat[article]{title}{\textit{\mkbibquote{#1}}}
\renewbibmacro{in:}{}
%\AtBeginBibliography{%
%    \urlstyle{rm}%
%}


%##############################################################################
%=== MISCELLANEOUS DOCUMENT SPECIFICATIONS ====================================
%##############################################################################
% Fewer overfull lines - used in the memoir class and allows a setting somewhere between \fussy and \sloppy
\midsloppy

% Tell memoir to implement the above
%\checkandfixthelayout

% Style for quotes
\newtcolorbox{myquote}{
    colback=green!3!white,
    colframe=green!30!white,
    grow to right by=-5mm,
    grow to left by=-5mm,
    boxrule=1pt,
    boxsep=0pt,
    breakable
} \makeatletter
\renewenvironment{quote}{\begin{myquote}}{\end{myquote}}

% Style for reminders
\newtcolorbox{reminder}{
    title=Lembre-se,
    colback=SkyBlue!30!white,
    colframe=SkyBlue!70!white,
    coltitle=SkyBlue!50!black,
    fonttitle=\sffamily\bfseries,
    boxrule=1pt,
    breakable
} \makeatletter

% Style for tips
\newtcolorbox{tip}{
    title=Dica,
    colback=LimeGreen!10!white,
    colframe=LimeGreen!40!white,
    coltitle=LimeGreen!60!black,
    fonttitle=\sffamily\bfseries,
    boxrule=1pt,
    breakable
} \makeatletter

% Style for warnings
\newtcolorbox{warning}{
    title=Aviso,
    colback=Maroon!10!white,
    colframe=Maroon!30!white,
    coltitle=Maroon!90!black!80!white,
    fonttitle=\sffamily\bfseries,
    boxrule=1pt,
    breakable
} \makeatletter


%##############################################################################
%=== VERBATIM, INLINE CODE AND CODE BLOCKS ====================================
%##############################################################################
% --- Colors for code styles --------------------------------------------------
\definecolor{solarized-light-bg}{HTML}{fef6e4}
\definecolor{solarized-light-bg-darker}{HTML}{eee6d4}
\definecolor{terminal-bgcolor}{RGB}{38,50,56}
\definecolor{terminal-bgcolor-title}{RGB}{48,60,66}
\definecolor{prompt-exitcode-fg}{HTML}{870000}
\definecolor{prompt-hostname-bg}{HTML}{afffff}
\definecolor{prompt-hostname-fg}{HTML}{008787}
\definecolor{prompt-username-bg}{HTML}{008787}
\definecolor{prompt-username-fg}{HTML}{eeeeee}
\definecolor{prompt-condaenv-bg}{HTML}{5faf87}
\definecolor{prompt-condaenv-fg}{HTML}{eeeeee}
\definecolor{prompt-time-bg}{HTML}{3a3a3a}
\definecolor{prompt-time-fg}{HTML}{a8a8a8}
\definecolor{prompt-git-bg}{HTML}{4e4e4e}
\definecolor{prompt-git-fg}{HTML}{bcbcbc}
\definecolor{prompt-workdir-bg}{HTML}{3a3a3a}
\definecolor{prompt-workdir-fg}{HTML}{00afaf}
\definecolor{prompt-userprompt-fg}{HTML}{767676}

% --- Standard styles for code blocks -----------------------------------------
\setminted{
    linenos,
    breaklines,
    fontsize=\small,
    style=solarized-light,
    bgcolor=solarized-light-bg,
    texcl=true,
    escapeinside=||
}
\setmintedinline{
    breaklines,
    fontsize=\small,
    style=solarized-light,
    bgcolor=SpringGreen!12!white,
    texcl=true,
    escapeinside=||
}

\newtcblisting{code}[2][]{%
    listing engine=minted,%
    colback=solarized-light-bg,%
    colframe=solarized-light-bg,%
    % colhighlight=solarized-light-bg-darker,%
    listing only,%
    minted style=solarized-light,%
    minted language=#2,%
    minted options={%
        autogobble, bgcolor=solarized-light-bg, breaklines, fontsize=\small,%
        linenos=true, numbersep=3mm, texcl=true, escapeinside=||, #1%
    },%
    left=5mm,%
    enhanced,%
    overlay={%
        \begin{tcbclipinterior}
            \fill[solarized-light-bg!95!black] (frame.south west) rectangle ([xshift=5mm]frame.north west);
        \end{tcbclipinterior}
    }%
}
% Style of the line numbers
% From https://tex.stackexchange.com/a/585711
\renewcommand{\theFancyVerbLine}{\ttfamily
    \textcolor{gray}{\footnotesize{\arabic{FancyVerbLine}}}}

% --- Terminal style ----------------------------------------------------------
% From: https://tex.stackexchange.com/a/609103
\tcbuselibrary{skins, breakable, breakable}
\tcbuselibrary{minted}

\newtcblisting{terminal}{
    listing engine=minted,
    colback=terminal-bgcolor,
    colframe=terminal-bgcolor,
    coltext=white,
    colbacktitle=terminal-bgcolor-title,
    listing only,
    minted style=native,
    minted language=text,
    minted options={
        autogobble, bgcolor=terminal-bgcolor, breaklines, fontsize=\small,
        linenos=false, texcl=true, escapeinside=||
    },
    enhanced,
}

\newtcblisting{cdjpasterminal}{
    listing engine=minted,
    colback=terminal-bgcolor,
    colframe=terminal-bgcolor,
    coltext=white,
    colbacktitle=terminal-bgcolor-title,
    listing only,
    minted style=native,
    minted language=text,
    minted options={
        autogobble, bgcolor=terminal-bgcolor, breaklines, fontsize=\small,
        linenos=false, texcl=true, escapeinside=||
    },
    enhanced,
    adjusted title=flush right,
    halign title=flush right,
    title=Terminal~~~~~~~~~~~~~~~~~~~~~~~~~~~~~~~~~~~~~~~~~~~~~~~~~~~~~~ \tikz{
        % Minimize, restore and maximize buttons based on the Breeze theme
        % Minimize
        \coordinate (coordmin1) at (0.15, 0.075);
        \coordinate (coordmin2) at (0, -0.075);
        \coordinate (coordmin3) at (-0.15, 0.075);
        \draw[lightgray] (coordmin1) -- (coordmin2) -- (coordmin3);
        % Restore
        \coordinate (coordres1) at (0.35, 0);
        \coordinate (coordres2) at (0.5, 0.15);
        \coordinate (coordres3) at (0.65, 0);
        \coordinate (coordres4) at (0.5, -0.15);
        \draw[lightgray] (coordres1) -- (coordres2) -- (coordres3)
                      -- (coordres4) -- (coordres1);
        % Maximize
        \coordinate (coordmax1) at (0.85, 0.15);
        \coordinate (coordmax2) at (1.15, -0.15);
        \coordinate (coordmax3) at (0.85, -0.15);
        \coordinate (coordmax4) at (1.15, 0.15);
        \draw[lightgray] (coordmax1) -- (coordmax2);  % \
        \draw[lightgray] (coordmax3) -- (coordmax4);  % /
    }
}

% --- The Prompt --------------------------------------------------------------
\newcommand{\PromptExitCode}[1]{%
    \color{prompt-exitcode-fg}{#1}%
}
\newcommand{\PromptHostName}[1]{%
    \colorbox{prompt-hostname-bg}{\color{prompt-hostname-fg}{#1}}%
}
\newcommand{\PromptUserName}[1]{%
    \colorbox{prompt-username-bg}{\color{prompt-username-fg}{#1}}%
}
\newcommand{\PromptCondaEnv}[1]{%
    \colorbox{prompt-condaenv-bg}{\color{prompt-condaenv-fg}{#1}}%
}
\newcommand{\PromptTime}[1]{%
    \colorbox{prompt-time-bg}{\color{prompt-time-fg}{#1}}%
}
\newcommand{\PromptGit}[1]{%
    \colorbox{prompt-git-bg}{\color{prompt-git-fg}{#1}}%
}
\newcommand{\PromptWorkDir}[1]{%
    \colorbox{prompt-workdir-bg}{\color{prompt-workdir-fg}{#1}}%
}
\newcommand{\PromptUserPrompt}[1]{%
    \color{prompt-userprompt-fg}{#1}%
}
\NewDocumentCommand{\prompt}{ommomomm}{%
    \IfValueT{#1}{\PromptExitCode{\vphantom{cdjpas}\Verb|#1 |}}%
    \PromptHostName{\vphantom{cdjpas}\Verb| #2 |}%
    \PromptUserName{\vphantom{cdjpas}\Verb| #3 |}%
    \IfValueT{#4}{\PromptCondaEnv{\vphantom{cdjpas}\Verb| #4 |}}%
    \PromptTime{\vphantom{cdjpas}\Verb| #5 |}%
    \IfValueT{#6}{\PromptGit{\vphantom{cdjpas}\Verb| #6 |}}%
    \PromptWorkDir{\vphantom{cdjpas}\Verb| #7 |}\\%
    \noindent\PromptUserPrompt{\vphantom{cdjpas}\Verb|#8|}%
}
