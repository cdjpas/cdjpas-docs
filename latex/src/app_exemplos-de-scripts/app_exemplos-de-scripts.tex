\chapter{Exemplos de Scripts}
\label{app:exemplos-de-scripts}
% Exemplos de scripts de submissão de jobs, scripts de configuração de ambiente, etc.

\lipsum[1-2]
